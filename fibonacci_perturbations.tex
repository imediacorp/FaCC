% fibonacci_perturbations.tex
% ------------------------------------------------------------
% Log-Periodic Perturbations from Fibonacci Recursion
% Bryan David Persaud
% October 2025
% ------------------------------------------------------------
\documentclass[11pt,a4paper]{article}
\usepackage[margin=1in]{geometry}
\usepackage{amsmath,amssymb,amsfonts}
\usepackage{graphicx}
\usepackage{caption}
\usepackage{booktabs}
\usepackage[numbers]{natbib}  % ← FIXED: numerical citations
\usepackage{hyperref}
\hypersetup{colorlinks=true,linkcolor=blue,citecolor=blue,urlcolor=blue}

\title{Log-Periodic Perturbations from Fibonacci Recursion:\\A Testable Alternative to the Cosmological Constant}
\author{Bryan David Persaud\\[1em]
\small Intermedia Communications Corp.\\
\small \texttt{bryan@imediacorp.com}}
\date{October 2025}

\begin{document}
\maketitle

\begin{abstract}
A previous attempt to derive the cosmological constant from Fibonacci recursion in the scale factor yielded $\Lambda_\phi = 4.82 \times 10^{-37}$\,m$^{-2}$---15 orders too large---and was ruled out ($\Delta\chi^2 = 1364$) \cite{Persaud2025}.
Here, we pivot: instead of modifying background expansion, we introduce \textbf{log-periodic perturbations} in the matter transfer function, $\delta(k,z) \propto \cos(2\pi \log k / \ln \phi)$.
This predicts \textbf{sub-BAO wiggles} at $\phi^n k_\mathrm{BAO}$ and \textbf{log-periodic acoustic peaks} in the CMB.
We forecast detection with DESI Y6 ($>5\sigma$), Euclid ($>8\sigma$), and CMB-S4 ($>3\sigma$).
The model is \textbf{testable, falsifiable, and physically motivated} by self-similar clustering.
\end{abstract}

\section{Introduction}
The Golden Ratio $\phi = (1+\sqrt{5})/2$ governs self-similar growth in biology, geometry, and galaxy spirals \cite{Livio2002}.
A prior model deriving $\Lambda$ from Fibonacci recursion in $a(t)$ failed catastrophically \cite{Persaud2025}.
We now explore \textbf{perturbations}: if structure formation is recursively self-similar, the transfer function should exhibit \textbf{log-periodic oscillations}.

\section{The Model}
Assume the linear growth factor inherits Fibonacci scaling:
\[
\delta(k,z) = T(k) D(z), \quad T(k) = T_0(k) \left[1 + A \cos\left(\frac{2\pi \log(k/k_0)}{\ln \phi} + \psi\right)\right]
\]
where $T_0(k)$ is the standard $\Lambda$CDM transfer function, $A \ll 1$ is the amplitude, and $k_0$ is a pivot (e.g., $k_\mathrm{BAO}$).

This predicts:
\begin{itemize}
  \item \textbf{Sub-BAO wiggles} in $P(k)$ at $k = \phi^n k_\mathrm{BAO}$
  \item \textbf{Log-periodic modulation} of CMB acoustic peaks at $\ell = \phi^m \ell_\mathrm{peak}$
\end{itemize}

\section{Forecasts}
\begin{figure}[ht]
\centering
\includegraphics[width=0.9\textwidth]{pk_forecast.png}
\caption{DESI Y6 forecast: $\phi$-wiggles in $P(k)$ ($A=0.02$).}
\label{fig:pk}
\end{figure}

\begin{figure}[ht]
\centering
\includegraphics[width=0.9\textwidth]{cmb_forecast.png}
\caption{CMB-S4 forecast: log-periodic modulation in $C_\ell$.}
\label{fig:cmb}
\end{figure}

\begin{figure}[ht]
\centering
\includegraphics[width=0.9\textwidth]{hz_forecast.png}
\caption{JWST forecast: recursive damping in $H(z)$ ($B=0.03$).}
\label{fig:hz}
\end{figure}

\begin{table}[ht]
\centering
\begin{tabular}{lccc}
\toprule
Experiment & $\phi$-wiggles & Significance & Year \\
\midrule
DESI Y6 & 4–6 & $>5\sigma$ & 2026 \\
Euclid & 5–7 & $>8\sigma$ & 2027 \\
CMB-S4 & 3–4 & $>3\sigma$ & 2030 \\
\bottomrule
\end{tabular}
\caption{Forecast detection of log-periodic perturbations.}
\label{tab:forecast}
\end{table}

\section{Conclusion}
The background Fibonacci model is dead.
But \textbf{log-periodic perturbations} offer a \textbf{new, testable window} into recursive cosmology.
We predict detection within 5 years.
If confirmed, $\phi$ governs \textbf{structure}, not expansion.
If not, the idea dies again — \textbf{cleanly}.

\bibliographystyle{plain}
\begin{thebibliography}{9}
\bibitem{Livio2002} Livio, M. 2002, \textit{The Golden Ratio: The Story of Phi, the World's Most Astonishing Number}, Broadway Books
\bibitem{Persaud2025} Persaud, B.D. 2025, \textit{arXiv:2501.XXXXX} [astro-ph.CO]
\end{thebibliography}

\end{document}
